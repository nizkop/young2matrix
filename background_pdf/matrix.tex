\documentclass[12pt,a4paper]{article}
\usepackage[utf8]{inputenc} 
\usepackage[ngerman]{babel}
\usepackage[T1]{fontenc}
\usepackage{amsmath}
\usepackage{amsfonts}
\usepackage{amssymb}
\usepackage{graphicx}
\usepackage[left=2cm,right=2cm,top=2cm,bottom=2cm]{geometry}
\usepackage{physics}
\usepackage{float}
\usepackage{multirow}
\setlength{\parindent}{0pt}
\usepackage{enumitem}
\usepackage{xcolor}
\usepackage{cancel} 

\newcommand{\h}[2]{\color{#1} #2 \color{black} }

\newcommand{\equalInM}[1]{\h{blue}{#1}} % umfasst mehr; inkl. Vorzeichen gleich über Sym-Anti und Anti-Sym und über die versch. Tableaus dieser Kategorie
\newcommand{\equalInTableau}[1]{\h{magenta}{#1}} % Vorzeichen vertauscht zw. den versch. Tableaus, aber Vorzeichen gleich bzgl. Anti-Sym zu Sym-Anti-Tausch
\newcommand{\equalAntiSym}[1]{\h{brown}{#1}} % zwar nicht in allen Tableaus einer M Sorte, aber gleich (inkl. Vorzeichen) bzgl. der Anti-Sym und der Sym-Anti Rechnung



\usepackage{soul}   % Für das Hervorheben von Text


\begin{document}

\tableofcontents

\section*{Index}
Variablen beziehen sich auf...
\begin{itemize}
\item röm. Buchstaben = Hilfsparameter
\item gr. Großbuchstaben = Hamiltonmatrix
\item gr. Kleinbuchstaben = Überlappung
\item $\Phi$, $\phi$ = Raumorbital
\item $\sigma$, $\Sigma$ = Spin
\item $\Sigma = \sigma$, daher wird nur die kleingeschriebene Variante verwendet
\end{itemize}
\newpage 


\section{Matrixelemente}

\subsection{Spin}

$\sigma_{11} = \bra{\left|
  \begin{array}{c|c}
  \hline 
    1 & 3 \\ \hline 
    2 & 4 \\
    \hline 
  \end{array}
\right|_\sigma} \ket{\left|
  \begin{array}{c|c}
  \hline 
    1 & 3 \\ \hline 
    2 & 4 \\
    \hline 
  \end{array}
\right|_\sigma} = 1$ \\ \\

$\sigma_{12} = \sigma_{21} = \bra{\left|
  \begin{array}{c|c}
  \hline 
    1 & 3 \\ \hline 
    2 & 4 \\
    \hline 
  \end{array}
\right|_\sigma }  \ket{\left|
  \begin{array}{c|c}
  \hline 
    1 & 2 \\ \hline 
    3 & 4 \\
    \hline 
  \end{array}
 \right|_\sigma} = \frac{1}{2}$ \\ \\ 

$\sigma_{22} = \bra{\left|
  \begin{array}{c|c}
  \hline 
    1 & 2 \\ \hline 
    3 & 4 \\
    \hline 
  \end{array}
\right|_\sigma }  \ket{\left|
  \begin{array}{c|c}
  \hline 
    1 & 2 \\ \hline 
    3 & 4 \\
    \hline 
  \end{array}
 \right|_\sigma} = 1$ \\

\subsection{Ortsanteile}
\subsubsection{Hamiltonmatrixelemente $\Phi$}
(s. 1., dann )
\begin{gather}
\Phi_{11} = \bra{
\left|
  \begin{array}{c|c}
  \hline 
    1 & 2 \\ \hline 
    3 & 4 \\
    \hline 
  \end{array}
\right| _{\Phi}
} \hat{H} \ket{
\left|
  \begin{array}{c|c}
  \hline 
    1 & 2 \\ \hline 
    3 & 4 \\
    \hline 
  \end{array}
\right| _{\Phi}
} \\
 = + D - \frac{1}{2} \cdot (ac|ca) - \frac{1}{2} \cdot (bd|db) + 1 \cdot (ab|ba) + 1 \cdot (cd|dc) - \frac{1}{2} \cdot (bc|cb) - \frac{1}{2} \cdot (ad|da) \\ 
 := D + A' \label{eq:def_A_Strich}
 \end{gather} \\


\begin{gather}
\Phi_{12} = \bra{
\left|
  \begin{array}{c|c}
  \hline 
    1 & 2 \\ \hline 
    3 & 4 \\
    \hline 
  \end{array}
\right| _{\Phi}
} \hat{H} \ket{
\left|
  \begin{array}{c|c}
  \hline 
    1 & 3 \\ \hline 
    2 & 4 \\
    \hline 
  \end{array}
\right| _{\Phi}
} \\
 = - \frac{1}{4} \cdot D - \frac{1}{4} \cdot (ab|ba) - \frac{1}{4} \cdot (cd|dc) - \frac{1}{4} \cdot (ac|ca) -  \frac{1}{4} \cdot (bd|db) + \frac{1}{2} \cdot (bc|cb) + \frac{1}{2} \cdot (ad|da) \\
 \label{eq:def_B_Strich}
 := -\frac{1}{4}\cdot D + B' 
 \end{gather} \\



\begin{gather}
\Phi_{22} = \bra{
\left|
  \begin{array}{c|c}
  \hline 
    1 & 3 \\ \hline 
    2 & 4 \\
    \hline 
  \end{array}
\right| _{\Phi}
} \hat{H} \ket{
\left|
  \begin{array}{c|c}
  \hline 
    1 & 3 \\ \hline 
    2 & 4 \\
    \hline 
  \end{array}
\right| _{\Phi}
} \\
= + D -  \frac{1}{2} \cdot (ab|ba) - \frac{1}{2} \cdot (cd|dc) + 1 \cdot (ac|ca) + 1 \cdot (bd|db) - \frac{1}{2} \cdot (bc|cb) - \frac{1}{2} \cdot (ad|da)
\\
\label{eq:def_C_Strich}
:= D + C' 
\end{gather} \\



\subsubsection{Überlappungselemente $\phi$}
$S^\Phi_{ii} = 1$ (s. gl. Funktionen treffen aufeinander), z.B.: 
$$S^\Phi_{11} = \bra{
\left|
  \begin{array}{c|c}
  \hline 
    1 & 2 \\ \hline 
    3 & 4 \\
    \hline 
  \end{array}
\right| _{\Phi}
} \ket{
\left|
  \begin{array}{c|c}
  \hline 
    1 & 2 \\ \hline 
    3 & 4 \\
    \hline 
  \end{array}
\right| _{\Phi}
} = 1 $$ \\


$$\phi_{12} = \bra{
\left|
  \begin{array}{c|c}
  \hline 
    1 & 2 \\ \hline 
    3 & 4 \\
    \hline 
  \end{array}
\right| _{\Phi}
}  \ket{
\left|
  \begin{array}{c|c}
  \hline 
    1 & 3 \\ \hline 
    2 & 4 \\
    \hline 
  \end{array}
\right| _{\Phi}
} $$

\tiny 
$$ \propto \bra{abcd - cbad - adcb + cdab + bacd - cabd - bdca + cdba
+ abdc - dbac - acdb + dcab + badc - dabc - bcda + dcba}$$
$$ \ket{abcd - bacd - abdc + badc + cbad - bcad - cbda + bcda 
+adcb - dacb - adbc + dabc + cdab - dcab - cdba + dcba}$$
\normalsize
$$= \bra{abcd}\ket{abcd} - \bra{cbad}\ket{cbad} - \bra{adcb}\ket{adcb} + \bra{cdab}\ket{cdab} $$
$$ - \bra{bacd}\ket{bacd}  - \bra{dcab}\ket{dcab}
- \bra{abdc}\ket{abdc} -  \bra{cdba}\ket{cdba}  $$
$$+ \bra{badc}\ket{badc} - \bra{bcda}\ket{bcda} - \bra{dabc}\ket{dabc} + \bra{dcba}\ket{dcba} = 4\cdot (+1) + 8 \cdot (-1) = -4 $$ mit Normierungsfaktor von je $\frac{1}{\sqrt{16}}$ folgt also: $\Phi_{12} = - \frac{1}{4} $ \\




$$\phi_{22} = \bra{
\left|
  \begin{array}{c|c}
  \hline 
    1 & 3 \\ \hline 
    2 & 4 \\
    \hline 
  \end{array}
\right| _{\Phi}
} \ket{
\left|
  \begin{array}{c|c}
  \hline 
    1 & 3 \\ \hline 
    2 & 4 \\
    \hline 
  \end{array}
\right| _{\Phi}
} = 1 $$ \\




\newpage 
\section{Matrix}

$$ 0 = \left|
  \begin{array}{ccc}
  H_{11} - S_{11} \cdot E && H_{12} - S_{12}\cdot E \\
  H_{12} - S_{12}\cdot E && H_{22} - S_{22} \cdot E 
  \end{array}
  \right|
  $$
  
1. Überlappungsintegrale $S = \phi \cdot \sigma$ \\

$S_{11} = \phi_{11} \cdot \sigma_{11} = \bra{
\left|
  \begin{array}{c|c}
  \hline 
    1 & 2 \\ \hline 
    3 & 4 \\
    \hline 
  \end{array}
\right| _{\Phi}
} \ket{
\left|
  \begin{array}{c|c}
  \hline 
    1 & 2 \\ \hline 
    3 & 4 \\
    \hline 
  \end{array}
\right| _{\Phi}
} \cdot \bra{\left|
  \begin{array}{c|c}
  \hline 
    1 & 3 \\ \hline 
    2 & 4 \\
    \hline 
  \end{array}
\right|_\sigma} \ket{\left|
  \begin{array}{c|c}
  \hline 
    1 & 3 \\ \hline 
    2 & 4 \\
    \hline 
  \end{array}
\right|_\sigma}
= 1 \cdot 1 = 1$ \\ 

$S_{12} = S_{21} = \phi_{12} \cdot \sigma_{12} = \phi_{12} \cdot \frac{1}{2} = - \frac{1}{4} \cdot \frac{1}{2} = - \frac{1}{8} $ \\ 

$S_{22} = 1$  \\ \\


2. Hamiltonmatrixelemente $H = \Phi \cdot \sigma$ :\\

$$H_{11} = \Phi_{11} \cdot \sigma_{11}= \bra{
\left|
  \begin{array}{c|c}
  \hline 
    1 & 2 \\ \hline 
    3 & 4 \\
    \hline 
  \end{array}
\right| _{\Phi}
} \hat{H} \ket{
\left|
  \begin{array}{c|c}
  \hline 
    1 & 2 \\ \hline 
    3 & 4 \\
    \hline 
  \end{array}
\right| _{\Phi}
} \cdot \bra{\left|
  \begin{array}{c|c}
  \hline 
    1 & 3 \\ \hline 
    2 & 4 \\
    \hline 
  \end{array}
\right|_\sigma} \ket{\left|
  \begin{array}{c|c}
  \hline 
    1 & 3 \\ \hline 
    2 & 4 \\
    \hline 
  \end{array}
\right|_\sigma} = \Phi_{11}\cdot 1 := D + A'
$$ \\


$$H_{12} = \Phi_{12} \cdot \sigma_{12}= \bra{
\left|
  \begin{array}{c|c}
  \hline 
    1 & 3 \\ \hline 
    2 & 4 \\
    \hline 
  \end{array}
\right| _{\Phi}
} \hat{H} \ket{
\left|
  \begin{array}{c|c}
  \hline 
    1 & 2 \\ \hline 
    3 & 4 \\
    \hline 
  \end{array}
\right| _{\Phi}
} \cdot \bra{\left|
  \begin{array}{c|c}
  \hline 
    1 & 2 \\ \hline 
    3 & 4 \\
    \hline 
  \end{array}
\right|_\sigma} \ket{\left|
  \begin{array}{c|c}
  \hline 
    1 & 3 \\ \hline 
    2 & 4 \\
    \hline 
  \end{array}
\right|_\sigma} = \Phi_{12}\cdot \frac{1}{2} $$ 
$$:= B \cdot \frac{1}{2} = \left( -\frac{1}{4}\cdot D + B' \right) \cdot \frac{1}{2} = - \frac{1}{8} \cdot D + \frac{1}{2} \cdot B'
$$



$$H_{22} = \Phi_{22} \cdot \sigma_{22}= \bra{
\left|
  \begin{array}{c|c}
  \hline 
    1 & 3 \\ \hline 
    2 & 4 \\
    \hline 
  \end{array}
\right| _{\Phi}
} \hat{H} \ket{
\left|
  \begin{array}{c|c}
  \hline 
    1 & 3 \\ \hline 
    2 & 4 \\
    \hline 
  \end{array}
\right| _{\Phi}
} \cdot \bra{\left|
  \begin{array}{c|c}
  \hline 
    1 & 2 \\ \hline 
    3 & 4 \\
    \hline 
  \end{array}
\right|_\sigma} \ket{\left|
  \begin{array}{c|c}
  \hline 
    1 & 2 \\ \hline 
    3 & 4 \\
    \hline 
  \end{array}
\right|_\sigma} = \Phi_{22}\cdot 1 = D + C'
$$


\subsection{Startgleichung}
Säkulargleichung: 
$$ 0 = \left|
  \begin{array}{ccc}
  H_{11} - S_{11} \cdot E && H_{12} - S_{12}\cdot E \\
  H_{12} - S_{12}\cdot E && H_{22} - S_{22} \cdot E 
  \end{array}
  \right|
  $$
  Spinanteile eingesetzt:
  $$ \Leftrightarrow  0 = \left|
  \begin{array}{ccc}
  A - 1 \cdot E && \frac{1}{2} \cdot B + \frac{1}{8}\cdot E \\
  \frac{1}{2} \cdot B + \frac{1}{8} \cdot E && C - 1 \cdot E 
  \end{array}
  \right|
  $$
  $$\Leftrightarrow 0 = \left|
  \begin{array}{ccc}
  \left( D+A' \right) -  1  \cdot E && \left( - \frac{1}{8} \cdot D + \frac{1}{2} \cdot B' \right) - \left( - \frac{1}{8} \right)\cdot E \\
 -\frac{1}{8}\cdot D + \frac{1}{2} \cdot B'
 + \frac{1}{8} \cdot E && \left(D+C' \right)  - 1  \cdot E 
  \end{array}
  \right|
  $$
  
  
\subsubsection{grober Ansatz}

 $$   = \left|
  \begin{array}{ccc}
  A - E && \frac{1}{2} \cdot B + \frac{1}{8}\cdot E \\
  \frac{1}{2} \cdot B + \frac{1}{8} \cdot E && C -  E 
  \end{array}
  \right|
  $$
  \begin{gather}
  \label{eq:grobGl}
  \Leftrightarrow 0 = \left( A-E\right) \cdot \left( C-E\right) 
  - \left( \frac{B}{2} + \frac{E}{8}\right) ^2 
  \end{gather} \\

  
\subsubsection{Ansatz mit D}
  $$ 0 = \left|
  \begin{array}{ccc}
 D+A'  -  E &&  -\frac{1}{8}\cdot D + \frac{1}{2} \cdot B'
 + \frac{1}{8} \cdot E\\
 -\frac{1}{8}\cdot D + \frac{1}{2} \cdot B'
 + \frac{1}{8} \cdot E && D+C'   -  E 
  \end{array}
  \right|
  $$
  $$ \Leftrightarrow 0 = 
  \left( D+A'-E\right) \cdot \left( D+C'-E\right) 
  - \left( -\frac{1}{8}\cdot D + \frac{1}{2} \cdot B'
 + \frac{1}{8} \cdot E  \right) ^2 
  $$


\subsection{Lösungsversuche}

\subsubsection{grob}\label{sec:grobeLsg} 
Gl. \eqref{eq:grobGl} lautete:
\begin{gather}
\tag{\ref{eq:grobGl}}
 0 = \left( A-E\right) \cdot \left( C-E\right) 
  - \left( \frac{B}{2} + \frac{E}{8}\right) ^2 
 \end{gather}\\
  
Auflösen der Klammern, um gl. Variablen zusammenfassen zu können: 
\begin{gather}
0 = A\cdot C - A\cdot E - E\cdot C + E^2 - \left(  \frac{B^2}{4} + 2\cdot \frac{B}{2} \cdot \frac{E}{8} + \frac{E^2}{64}  \right)
\end{gather}
\begin{gather}
\Leftrightarrow 0 =A\cdot C - E\cdot \left(A-C\right) + E^2 - \frac{B^2}{4} - \frac{B \cdot E}{8} - \frac{E^2}{64}
\end{gather}

sortieren nach Termen von E: 
$$\Leftrightarrow 0 = E^2\cdot \left( 1-\frac{1}{64}\right) + E \cdot \left( C-A - \frac{B}{8}\right) + A \cdot C - \frac{B^2}{4} $$
 $$\Leftrightarrow 0 = E^2\cdot \frac{63}{64} + E \cdot \left( C-A - \frac{B}{8}\right) + A \cdot C - \frac{B^2}{4} $$
 Teilen durch Vorfaktor von $E^2$: 
$$\Leftrightarrow 0 = E ^2 + E \cdot \underbrace{ \frac{64}{63}\cdot \left( C-A - \frac{B}{8}\right)}_p + \underbrace{\frac{64}{63} \cdot \left( A\cdot C - \frac{B^2}{4} \right) }_q$$
 pq-Formel, um die Werte von $E$ zu erhalten: 
 $$\Leftrightarrow E = \frac{1}{2}\cdot \frac{64}{63}\cdot \left( C-A - \frac{B}{8}\right) \pm \sqrt{ \left( \frac{32}{63}\cdot \left( C-A - \frac{B}{8}\right) \right) ^2 
 - \frac{64}{63} \cdot \left( A\cdot C - \frac{B^2}{4} \right)
 }
  $$
  $$\Leftrightarrow E =  \frac{32}{63}\cdot \left( C-A - \frac{B}{8}\right) \pm 
  \sqrt{ \frac{32^2}{63^2}\cdot 
\left( C-A - \frac{B}{8}\right) ^2 
 - \frac{64}{63} \cdot  A\cdot C + \frac{16}{63} \cdot B^2 
 }
  $$
    $$\Leftrightarrow E =  \frac{32}{63}\cdot \left( C-A - \frac{B}{8}\right) \pm 
  \sqrt{
  \frac{16}{63} \cdot \left( 
     \frac{64}{63}\cdot 
\left( C-A - \frac{B}{8}\right) ^2 
 - 4 \cdot  A\cdot C + B^2 
 \right)
 }
  $$
  $$\Leftrightarrow E =  \frac{32}{63}\cdot \left( C-A - \frac{B}{8}\right) \pm 
    \frac{4}{\sqrt{63} }
  \cdot 
  \sqrt{
     \frac{64}{63}\cdot 
\left( C-A - \frac{B}{8}\right) ^2 
 - 4 \cdot  A\cdot C + B^2 
 }
  $$ \\
  
  Betrachtung des Terms unter der Wurzel, um die quadrierte Klammer aufzulösen:   
$$ \frac{64}{63}\cdot \left( C-A - \frac{B}{8}\right) ^2 - 4 \cdot  A\cdot C + B^2 $$
$$= \frac{64}{63}\cdot \left( C-A - \frac{B}{8}\right)\cdot \left( C-A - \frac{B}{8}\right) - 4 \cdot  A\cdot C + B^2 $$
$$ \frac{64}{63}\cdot \left( 
C ^2 - C \cdot A - C \cdot \frac{B}{8}
- A \cdot C + A ^2 + A \cdot \frac{B}{8}
- \frac{B}{8}\cdot C + \frac{B}{8}\cdot A + \frac{B^ 2}{64} 
\right) - 4 \cdot  A\cdot C + B^2 $$
$$= \frac{64}{63}\cdot \left( 
C ^2 + A ^2 + \frac{B^ 2}{64} 
- 2\cdot C \cdot A - C \cdot \frac{B}{4}
 + A \cdot \frac{B}{4}
\right) - 4 \cdot  A\cdot C + B^2 $$
zusammenfassen gl. Terme der Variablen:
$$= A^2 \cdot \frac{64}{63} 
+ B^2 \cdot \left( \frac{1}{63} + 1\right) + C ^2 \cdot \frac{64}{63} 
+ A\cdot B \cdot \left(   \frac{64}{63} \cdot \frac{1}{4}\right) 
 + A\cdot C \cdot\left( -2 \cdot \frac{64}{63} - 4 \right) 
 + B\cdot C \cdot \left(   \frac{64}{63} \cdot \frac{1}{4}\right) 
$$
$$= \left( A^2 + C ^2\right)\cdot \frac{64}{63} 
+ B^2 \cdot \frac{65}{63} 
+ \left( A+C\right) \cdot B \cdot\frac{16}{63} 
 + A\cdot C \cdot \frac{-380}{63} 
$$
$$= 
\frac{1}{63} \cdot \left(
\left( A^2 + C ^2\right)\cdot 64
+ B^2 \cdot 65 
+ \left( A+C\right) \cdot B \cdot 16
 - A\cdot C \cdot 380
 \right) 
$$
d.h. aus $E$ wird demnach: 
$$ E = \frac{32}{63}\cdot \left( C-A - \frac{B}{8}\right) \pm 
    \frac{4}{\sqrt{63} }
  \cdot 
  \sqrt{\frac{1}{63}}\cdot 
  \sqrt{
     \left( A^2 + C ^2\right)\cdot 64
+ B^2 \cdot 65 
+ \left( A+C\right) \cdot B \cdot 16
 - A\cdot C \cdot 380
 }$$
 \begin{gather} \label{eq:E_ABC}
  \Leftrightarrow E = \frac{32}{63}\cdot \left( C-A - \frac{B}{8}\right) \pm 
    \frac{4}{63}
  \cdot 
  \sqrt{
     \left( A^2 + C ^2\right)\cdot 64
+ B^2 \cdot 65 
+ \left( A+C\right) \cdot B \cdot 16
 - A\cdot C \cdot 380
 }
 \end{gather}


\subsubsection{Einsetzen der D-Terme für A und C} \label{sec:einsetzenAC}
aus \ref{sec:grobeLsg} kann Genaueres geschlussfolgert werden, wenn $A$, $B$ und $C$ konkreter formuliert werden. Wie oben gilt dabei: 
\begin{gather}
\label{eq:ABC_strich}
 A= D+A' , \qquad B = - \frac{1}{4} \cdot D + B', \qquad C = D+C'
 \end{gather}
(da  $A = \Phi_{11} = H_{11}$, $B = \Phi_{12}$,$C = \Phi_{22} = H_{22}$)


Zur Übersichtlichkeit wird $E$ (gemäß \eqref{eq:E_ABC}) folgenderweise in Teilterme zerlegt:
 $$ E = \underbrace{\frac{32}{63}\cdot \left( C-A - \frac{B}{8}\right) }_{a} \pm 
    \frac{4}{63}
  \cdot 
  \sqrt{
  \underbrace{
\underbrace{
     \left( A^2 + C ^2\right)\cdot 64}_c
+ B^2 \cdot 65 
\underbrace{ + \left( A+C\right) \cdot B \cdot 16}_d
\underbrace{ - A\cdot C \cdot 380}_e
 }_b}$$
\begin{gather}
\label{eq:E_in_abc}
\Leftrightarrow E = a \pm \frac{4}{63} \cdot \sqrt{b} 
= a \pm \frac{4}{63} \cdot \sqrt{c + B ^2 \cdot 65 + d + e} 
\end{gather}

Mit diesen Teiltermen sowie den Ausdrücken für $A$ und $C$ folgt: 
\begin{itemize}
\item $a = \frac{32}{63}\cdot \left( C-A - \frac{B}{8}\right)  = \frac{32}{63}\cdot \left( \left( D+C'\right) - \left( D+A'\right) - \frac{B}{8}\right) $
\begin{gather}
 = \frac{32}{63}\cdot \left( C' - A' 
- \frac{B}{8}\right) 
\label{eq:a_ACeingesetzt}
\end{gather}

\item $b= c + B ^2 \cdot 65 + d + e$
\begin{itemize}
\item[*] $c = \left( A^2 + C ^2\right)\cdot 64 $
$$ =64\cdot  \left( \left( D+A'\right)^2 +  \left( D+C'\right) ^2\right) $$
$$ = 64\cdot\left( \left( D^2 + 2 \cdot D \cdot A' +{A'}^2\right) +  
\left( D^2 + 2 \cdot D \cdot C' +C'^2\right) \right) $$
$$ = 64\cdot\left( 2 \cdot D^2 + 2 \cdot D \cdot \left( A' + C'\right) + A'^2 + C'^2
 \right) $$
 $$ = 128\cdot D^2 + 128 \cdot D \cdot \left( A' + C'\right) + 64\cdot \left( A'^2 + C'^2
 \right) $$
 
\item[*] $d =  \left( A+C\right) \cdot B \cdot 16$ 
$$ =16 B \cdot \left( \left( D+A'\right)+\left( D+C'\right)\right)$$ 
$$ =16 B \cdot \left(
2 \cdot D + A' + C' 
\right)$$ 
$$ = 16 B \cdot 
2 \cdot D + 16 B \cdot \left( A' + C' 
\right)$$ 

\item[*] $e=  - A\cdot C \cdot 380 $ 
$$ =  - 380\cdot \left( D+A'\right)\cdot \left( D+C'\right) $$
$$ =  - 380\cdot \left( 
D ^2 + A' \cdot D + C' \cdot D + A' \cdot C' 
\right) $$
$$ =  - 380\cdot \left( 
D ^2 + D \cdot \left( A' + C' \right) + A' \cdot C' 
\right) $$
$$ =  - 380\cdot 
D ^2 - 380\cdot D   \left( A' + C' \right) 
-380\cdot A' \cdot C' $$
\end{itemize}
\item im zusammengesetzten $b$ können die Terme von $D$ zusammengefasst werden: 
$$b= c + B ^2 \cdot 65 + d + e$$
$$=
 \left( 128\cdot D^2 + 128 \cdot D \cdot \left( A' + C'\right) + 64\cdot \left( A'^2 + C'^2 \right)\right)  $$
 $$ + B ^2 \cdot 65
+ \left(16 B \cdot 2 \cdot D + 16 B \cdot \left( A' + C' \right) \right) $$
$$
+ \left(  - 380\cdot D ^2 - 380\cdot D   \left( A' + C' \right) -380\cdot A' \cdot C'\right)$$

\begin{gather}
\label{eq:b_zahlen}
 =
 D ^2 \cdot \left(128-380\right) \\
 + D \cdot \left(
 128\cdot \left( A' + C' \right) + 32 \cdot B - 380 \cdot \left( A' + C'\right) 
  \right) \notag \\ \notag
 + \left( 
 64\cdot \left(A'^2 + C'^2\right) +65\cdot B^2 +
 16\cdot B\cdot \left(A' + C'\right) - 380 \cdot A' \cdot C' 
 \right) 
\end{gather}

\begin{gather}
\label{eq:b_summanden} 
= -252\cdot D ^2 + D \cdot \left(
 -252\cdot \left( A' + C' \right) + 32 \cdot B 
  \right) \\ \notag
 + \left( 
 64\cdot \left(A'^2 + C'^2\right) +65\cdot B^2 +
 16\cdot B\cdot \left(A' + C'\right) - 380 \cdot A' \cdot C' 
 \right) 
\end{gather}

\begin{gather}
 = -252\cdot D ^2 + 4D \cdot \left(8 \cdot B 
 -63\cdot \left( A' + C' \right) 
  \right)  \label{eq:b_nachD}\\ \notag
 + 
 64\cdot \left(A'^2 + C'^2\right) +65\cdot B^2 +
 16\cdot B\cdot \left(A' + C'\right) - 380 \cdot A' \cdot C' 
\end{gather}



\end{itemize}
  
Damit ist $E$ nicht wirklich vereinfacht: 

\begin{gather}
\label{eq:e_nachACStrich}
 E = 
\frac{32}{63}\cdot \left( C' - A' 
- \frac{B}{8}\right)  \pm \frac{4}{63} \cdot
\\ \notag \sqrt{
-252\cdot D ^2 + 4D \cdot \left(8 \cdot B 
 -63\cdot \left( A' + C' \right) 
  \right) 
 + 
 64\cdot \left(A'^2 + C'^2\right) +65\cdot B^2 +
 16\cdot B\cdot \left(A' + C'\right) - 380 \cdot A' \cdot C' 
} 
\end{gather}



\subsubsection*{B im Wurzelterm b}

Im Term \eqref{eq:b_summanden} kommt auch $B$ in mehreren Potenzen vor, analog zu $D$. D.h. es könnte nach $D$ und $B$ sortiert werden. Es galt:

\begin{gather} 
b = 
 -252\cdot D ^2 + D \cdot \left(
 -252\cdot \left( A' + C' \right) + 32 \cdot B 
  \right) \notag \\ \notag
 + \left( 
 64\cdot \left(A'^2 + C'^2\right) +65\cdot B^2 +
 16\cdot B\cdot \left(A' + C'\right) - 380 \cdot A' \cdot C' 
 \right) 
\end{gather}
D.h. ebenso gilt: 
\begin{gather} 
b = 
\underbrace{-252\cdot D^2 }
+ \underbrace{23 \cdot B \cdot D}
+ \underbrace{65\cdot B ^2} \\ \notag
\underbrace{
-252\cdot D \cdot \left( A' + C'\right) 
+ 16 \cdot B \cdot \left(A' + C'\right) 
}
\underbrace{
+ 64\cdot \left( A'^2 + C'^2\right) - 380 \cdot A' \cdot C'
}
\end{gather}

\begin{gather} 
= 
\underbrace{-252\cdot D^2 }
+ \underbrace{23 \cdot B \cdot D}
+ \underbrace{65\cdot B ^2} \\ \notag
+ \underbrace{
4  \cdot \left( A' + C'\right) \cdot \left( -63\cdot D + 4\cdot B\right)
}
\underbrace{
+ 64\cdot \left( A'^2 + C'^2\right) - 380 \cdot A' \cdot C'
}
\end{gather}
Zum Anwenden der bionomischen Formel gibt es u.a. das Problem, dass $B$ und $D$ nicht nur als Produkt, sondern zudem auch noch als Summanden vorliegen. 

\subsubsection{B zu B'}
Nach \ref{sec:einsetzenAC} kann auch noch $B$ eingesetzt werden. Mit \eqref{eq:ABC_strich} folgt für \eqref{eq:b_nachD}: 

\begin{gather}
 b = -252\cdot D ^2 + 4D \cdot \left(8 \cdot \left( - \frac{1}{4} \cdot D + B'\right)
 -63\cdot \left( A' + C' \right) 
  \right)  \label{eq:}\\ \notag
 + 
 64\cdot \left(A'^2 + C'^2\right) +65\cdot \left( - \frac{1}{4} \cdot D + B'\right)^2 +
 16\cdot \left( - \frac{1}{4} \cdot D + B'\right)\cdot \left(A' + C'\right) - 380 \cdot A' \cdot C' 
 \\ \notag \\ 
   = -252\cdot D^2 + 4D \cdot 
   \left(8 \cdot B' - 2\cdot D \right) 
 -63\cdot \left( A' + C' \right) +64\cdot \left(A'^2 + C'^2 \right)
 \\ \notag
  +
  65\cdot \left(
 \frac{D^2}{16} - 2 \cdot \frac{1}{4} \cdot D \cdot B' + B'^2  
  \right) +
 16\cdot \left( 
 B' - \frac{1}{4} \cdot D
 \right)
 \cdot \left(A' + C'\right) - 380 \cdot A' \cdot C' 
\end{gather} 

\begin{gather}
= -252 \cdot D ^2  + D \cdot 32 \cdot B' - 8 \cdot D ^2 
-63\cdot \left( A' + C' \right) +64\cdot \left(A'^2 + C'^2 \right) \\
\notag
+ \frac{65}{16}\cdot D ^2 - \frac{65}{2}\cdot D \cdot B' + 65 \cdot B'^2 \\ \notag 
+ 16 \cdot B' \cdot \left(A' + C' \right) 
- 4 \cdot D \cdot \left(A' + C'\right) 
- 380 \cdot A' \cdot C' 
\end{gather}


Es kann erneut nach Potenzen von $D$ sortiert werden: 
\begin{gather}
b = 
D^2 \cdot  \left(
-252-8+\frac{65}{16}
 \right) 
\\ \notag
+ D \cdot \left(
32\cdot B' - \frac{65}{2} \cdot B' - 4\cdot \left(A' + C'\right) 
 \right) \\ \notag
+ \left( 
63\cdot \left(A' + C'\right) + 64 \cdot  \left(A'^2 + C'^2\right)
+ 65 \cdot B'^2 + 16 \cdot B' \cdot  \left(A' + C'\right)- 380 \cdot A' \cdot C'
\right)  \\ \notag \\ \label{eq:b_mitBStrich}
=  -\frac{4095}{16} \cdot D^2 
+ D\cdot \left( 
- \frac{B'}{2} - 4\cdot \left(A' + C'\right) 
\right) \\ \notag
+  
\left(A' + C'\right) \cdot \left( 63+ 16\cdot B'\right) 
+ 64 \cdot  \left(A'^2 + C'^2\right)
+ 65 \cdot B'^2 
- 380 \cdot A' \cdot C'
\end{gather} \\



\begin{gather}
\end{gather} \\


\subsection{A' und C'}
$E$ abhängig von Termen wie $A'^2+C'^2$, $A'+C'$ und $A' \cdot C'$. Nach \eqref{eq:def_A_Strich}, \eqref{eq:def_C_Strich} und \eqref{eq:def_B_Strich} gilt: 

\begin{gather}
A' = - \frac{1}{2} \cdot (ac|ca) - \frac{1}{2} \cdot (bd|db) + 1 \cdot (ab|ba) + 1 \cdot (cd|dc) - \frac{1}{2} \cdot (bc|cb) - \frac{1}{2} \cdot (ad|da) \\ 
C' =  -  \frac{1}{2} \cdot (ab|ba) - \frac{1}{2} \cdot (cd|dc) + 1 \cdot (ac|ca) + 1 \cdot (bd|db) - \frac{1}{2} \cdot (bc|cb) - \frac{1}{2} \cdot (ad|da) \\
 B' = - \frac{1}{4} \cdot (ab|ba) - \frac{1}{4} \cdot (cd|dc) - \frac{1}{4} \cdot (ac|ca) -  \frac{1}{4} \cdot (bd|db) + \frac{1}{2} \cdot (bc|cb) + \frac{1}{2} \cdot (ad|da) \label{eq:b_strich}
\end{gather} \\

D.h. es folgt: 
\begin{gather}
A' + C' = 
\frac{1}{2} (ac|ca) + \frac{1}{2} (bd|db) + \frac{1}{2} (ab|ba) + \frac{1}{2} (cd|dc) 
- 1 (bc|cb) -1 (ad|da)
\end{gather}
Der Vergleich mit \eqref{eq:b_strich} zeigt: 
\begin{gather}
\label{eq:ACStrich_zu_BStrich}
A' + C' = -2 \cdot B'
\end{gather}

\begin{gather}
C' -  A'  =  
- \frac{3}{2} \cdot (ab|ba) 
- \frac{3}{2} \cdot(cd|dc)
+ \frac{1}{2} \cdot(ac|ca) 
+ \frac{1}{2} \cdot(bd|db) 
- 1\cdot (bc|cb) - 1 \cdot(ad|da)
\end{gather}
 

\subsubsection{Einsetzen}
Einsetzen von \eqref{eq:ACStrich_zu_BStrich} in die Terme von \eqref{eq:b_mitBStrich}: 

Aus 
\begin{gather}
- \frac{B'}{2} - 4\cdot \left(A' + C'\right) 
\end{gather} wird
\begin{gather}
- \frac{B'}{2} - 4\cdot \left(- 2 \cdot B'\right)  
= - \frac{B'}{2} + 8\cdot B' = \frac{15}{2} \cdot B'
\end{gather} \\


Und aus 
\begin{gather}
\left(A' + C'\right) \cdot \left( 63+ 16\cdot B'\right) 
\end{gather} wird 
\begin{gather}
\left(-2 \cdot B'\right) \cdot \left( 63+ 16\cdot B'\right) 
= -2 \cdot B'\cdot \left( 63+ 16\cdot B'\right) \\ 
= - 126 \cdot B' - 32 \cdot B'^2 
\end{gather} \\


Daraus folgt für den Wurzelterm $b$ also: 
\begin{gather}
b = -\frac{4095}{16} \cdot D^2 
+ D\cdot \left( 
- \frac{B'}{2} - 4\cdot \left(A' + C'\right) 
\right) 
\tag{\ref{eq:b_mitBStrich}}
\\ \notag
+  
\left(A' + C'\right) \cdot \left( 63+ 16\cdot B'\right) 
+ 64 \cdot  \left(A'^2 + C'^2\right)
+ 65 \cdot B'^2 
- 380 \cdot A' \cdot C'
\\ \notag \\ 
= -\frac{4095}{16} \cdot D^2 
+ D\cdot \left( 
- \frac{15}{2} \cdot B'  
\right) 
\\ \notag
- 126 \cdot B' - 32 \cdot B'^2
+ 64 \cdot  \left(A'^2 + C'^2\right)
+ 65 \cdot B'^2 
- 380 \cdot A' \cdot C'
\end{gather} 

\begin{gather}
= -\frac{4095}{16} \cdot D^2 
- \frac{15}{2} \cdot D\cdot B'  
\\ \notag
+ 33\cdot B'^2 
-126 \cdot B' 
+ \left( 64 \cdot  \left(A'^2 + C'^2\right) - 380\cdot A' \cdot C' \right) 
\end{gather}

  
bzw. nach $B'$-Potenzen sortiert: 
\begin{gather}
b = 
 33\cdot B'^2 
 + B' \cdot \left( 
 - \frac{15}{2} \cdot D-126
 \right) \\ \notag
-\frac{4095}{16} \cdot D^2 
+ \left( 64 \cdot  \left(A'^2 + C'^2\right) - 380\cdot A' \cdot C' \right) \\
= 
 33\cdot B'^2 
 - \frac{3}{2} \cdot B' \cdot \left( 
 5 \cdot D + 168
 \right) \\ \notag
-\frac{4095}{16} \cdot D^2 
+ \left( 64 \cdot  \left(A'^2 + C'^2\right) - 380\cdot A' \cdot C' \right)
\end{gather} \\
  
  
Laut \eqref{eq:e_nachACStrich} ist zudem noch der Vorfaktor $a$ von $B$ abhängig. Einsetzen führt zu:
\begin{gather}
 a = \frac{32}{63}\cdot \left( C' - A' 
- \frac{B}{8}\right) 
\tag{\ref{eq:a_ACeingesetzt}} \\ \notag \\
 = \frac{32}{63}\cdot \left( C' - A' 
- \frac{1}{8} \cdot \left( - \frac{1}{4} \cdot D + B' \right) 
\right)  \\
 = \frac{32}{63}\cdot \left( C' - A'  + \frac{D}{16} - \frac{B'}{8}
 \right)
\end{gather}

Damit lautet $E$ also: 
\begin{gather}
\tag{\ref{eq:E_in_abc}}
E = a \pm \frac{4}{63} \cdot \sqrt{b} 
\\ 
= \frac{32}{63}\cdot \left( C' - A'  + \frac{D}{16} - \frac{B'}{8}
 \right) 
 \pm \frac{4}{63}  \cdot \\ \notag 
 \sqrt{ 33\cdot B'^2 
 - \frac{3}{2} \cdot B' \cdot \left( 
 5 \cdot D + 168
 \right)
-\frac{4095}{16} \cdot D^2 
+ 64 \cdot  \left(A'^2 + C'^2\right) - 380\cdot A' \cdot C' }
\end{gather}





\begin{gather}
\end{gather}

  
  
  
  
  
  
  
  
  
  
  
  

\end{document}