\documentclass[fleqn]{article}%
\usepackage[T1]{fontenc}%
\usepackage[utf8]{inputenc}%
\usepackage{lmodern}%
\usepackage{textcomp}%
\usepackage{lastpage}%
\usepackage[a4paper,left=2cm,right=2cm,top=2.5cm,bottom=2.5cm]{geometry}%
\usepackage[doublespacing]{setspace}%
\usepackage{fancyhdr}%
\usepackage{ragged2e}%
%
\usepackage[ngerman]{babel}%
\usepackage{tocloft}%
\renewcommand{\cftsecleader}{\cftdotfill{\cftdotsep}}   % activating dots in table of contents%
\usepackage{physics}%
\usepackage{breqn}%
\usepackage{needspace}%
\newcommand{\checkpagebreak}{\needspace{.25\textheight}}%
%
\begin{document}%
\normalsize%
\noindent%
\thispagestyle{empty}%
\begin{center}%
\vspace*{4cm}%
\Huge{}%
Erstellung %
\\ %
 von Überlappungs{-} und Hamiltonintegralen %
\\ %
 auf Basis der Symmetrieeigenschaften %
\\ %
 von Young{-}Tableaus%
\\ %
\vspace{1cm}%
\Large{hier für die Permutationsgruppe: 2}%
\\ %
\vspace{4cm}%
\Large{28. Juni 2024}%
\end{center}%
\newpage%
\setcounter{page}{1}%
\pagestyle{fancy}%
\fancyhf{}%
\fancyhead[L]{ Permutationsgruppe 2 }%
\fancyhead[R]{\nouppercase{\leftmark}}%
\renewcommand{\footrulewidth}{0.4pt}%
\fancyfoot[C]{\thepage}%
\section{Young{-}Tableaus}%
\label{sec:Young{-}Tableaus}%
Die möglichen (Standard-)Young-Tableaus zur Gruppe 2 lauten:

%
\vspace{0.25cm}%
\begin{dmath*}\left[2\right]:\quad\begin{array}{|c|c|} \hline 1 & 2\\ \cline{1-2} \end{array} \end{dmath*}%
\vspace{0.25cm}%
\begin{dmath*}\left[1^2\right]:\quad\begin{array}{|c|} \hline 1\\ \cline{1-1} 2\\ \cline{1-1} \end{array} \end{dmath*}%
\vspace{0.25cm}%
\newpage%
\section{Ausmultiplizierte Young{-}Tableaus}%
\label{sec:AusmultiplizierteYoung{-}Tableaus}%

%
\subsection{Raum{-}Funktionen}%
\label{subsec:Raum{-}Funktionen}%
$a, b, c, \hdots \quad $ = allgemeine Funktionen, die beispielsweise p-Orbitale repräsentieren könnten

%
\vspace{0.25cm}%
\begin{dmath*}\left[2\right]:\end{dmath*}%
\vspace{0.25cm}%
\begin{dmath*}\begin{array}{|c|c|} \hline 1 & 2\\ \cline{1-2} \end{array} \quad \frac{1}{\sqrt{2}} \left( + a_{1} \cdot b_{2}  + a_{2} \cdot b_{1}\right) \end{dmath*}%
\vspace{0.25cm}%
\vspace{0.25cm}%
\vspace{0.25cm}%
\begin{dmath*}\left[1^2\right]:\end{dmath*}%
\vspace{0.25cm}%
\begin{dmath*}\begin{array}{|c|} \hline 1\\ \cline{1-1} 2\\ \cline{1-1} \end{array} \quad \frac{1}{\sqrt{2}} \left( + a_{1} \cdot b_{2}  - a_{2} \cdot b_{1}\right) \end{dmath*}%
\vspace{0.25cm}%
\vspace{0.25cm}%
\checkpagebreak%
\subsection{Spin{-}Funktionen}%
\label{subsec:Spin{-}Funktionen}%
Die möglichen Kombinationen $\ket{S \; M_S}$ für die Tableaus der Permutationsgruppe 2 lauten:

%
\vspace{0.25cm}%
\begin{dmath*}\left[2\right]:\end{dmath*}%
\vspace{0.25cm}%
\begin{dmath*}\begin{array}{|c|c|} \hline 1 & 2\\ \cline{1-2} \end{array} \qquad \ket{ 1 \quad  +0 } = \frac{1}{\sqrt{2}} \left( + \alpha_{1} \cdot \beta_{2}  + \alpha_{2} \cdot \beta_{1}\right) \end{dmath*}%
\vspace{0.25cm}%
\begin{dmath*}\begin{array}{|c|c|} \hline 1 & 2\\ \cline{1-2} \end{array} \qquad \ket{ 1 \quad  +1 } = \left( + \alpha_{1} \cdot \alpha_{2}\right) \end{dmath*}%
\vspace{0.25cm}%
\begin{dmath*}\begin{array}{|c|c|} \hline 1 & 2\\ \cline{1-2} \end{array} \qquad \ket{ 1 \quad  -1 } = \left( + \beta_{1} \cdot \beta_{2}\right) \end{dmath*}%
\vspace{0.25cm}%
\vspace{0.25cm}%
\vspace{0.25cm}%
\begin{dmath*}\left[1^2\right]:\end{dmath*}%
\vspace{0.25cm}%
\begin{dmath*}\begin{array}{|c|} \hline 1\\ \cline{1-1} 2\\ \cline{1-1} \end{array} \qquad \ket{ 0 \quad  +0 } = \frac{1}{\sqrt{2}} \left( + \alpha_{1} \cdot \beta_{2}  - \alpha_{2} \cdot \beta_{1}\right) \end{dmath*}%
\vspace{0.25cm}%
\vspace{0.25cm}%
\newpage%
\section{Überlappungsintegrale}%
\label{sec:berlappungsintegrale}%

%
\subsection{Raumfunktionen}%
\label{subsec:Raumfunktionen}%
 (nur nicht verschwindende Kombinationen gezeigt)\\Identische Tableaus ergeben (aufgrund der normierten Funktionen darin) automatisch 1 und werden daher hier nicht aufgelistet.

%
\checkpagebreak%
\subsection{Spinfunktionen}%
\label{subsec:Spinfunktionen}%
(nur nicht verschwindende Kombinationen gezeigt)\\ Überlapp zw. versch. Tableaus ist 0 (wird hier ausgelassen), Überlapp zwischen gleichen Tableaus mit gleichem $m_S$-Wert ist 1 (wird hier ausgelassen)\\hier informale Darstellung der Tableaus mit Spinfunktionen nach dem Schema: 

%
\begin{dmath*}\bra{\,\text{Tableau 1}\,}\ket{\,\text{Tableau 2}\,} = \bra{\, \underbrace{S \quad m_S}_{\text{von Tableau 1}} \,} \ket{\, \underbrace{S \quad m_S}_{\text{von Tableau 2}} \,} = \underbrace{...}_{\text{Überlapp der Tableaus 1 und 2}}\end{dmath*}%
\vspace{0.25cm}%
\vspace{0.25cm}%
\vspace{0.25cm}%
\newpage%
\section{Hamiltonmatrixelemente}%
\label{sec:Hamiltonmatrixelemente}%

%
\subsection{Raum{-}Funktionen}%
\label{subsec:Raum{-}Funktionen}%

%
\vspace{0.25cm}%
\vspace{0.25cm}%
\begin{dmath*}\bra{\begin{array}{|c|c|} \hline 1 & 2\\ \cline{1-2} \end{array} }\hat{H}\ket{\begin{array}{|c|c|} \hline 1 & 2\\ \cline{1-2} \end{array} }_{\Phi} = + 2  \cdot \bra{ a_{1} \cdot b_{2} } \hat{H} \ket{ a_{1} \cdot b_{2}}\end{dmath*}%
\vspace{0.25cm}%
\checkpagebreak%
\subsection{Spin{-}Funktionen}%
\label{subsec:Spin{-}Funktionen}%
Achtung: Der Hamiltonoperator ist unabhängig vom Spin, daher werden die Hamiltonintegrale der Spin-Tableaus zu den Überlappungsintegralen und werden hier nicht erneut aufgeführt. (s. Kapitel 4.2) 

%
\newpage%
\tableofcontents%
\thispagestyle{fancy}%
\end{document}